%********************************************
%Weitere wichtige Pakete
\usepackage{amsmath} %Formeln
\usepackage{graphicx} %Graphikformate
\usepackage{flafter} %Float-Umgebungen erscheinen frühestens nach ihrer Position im Quellcode
\usepackage{pdflscape} %Querseiten
\usepackage{booktabs} % Schöne Unterteilungsstriche für Tabellen: \toprule, \bottomrule
\usepackage{pdfpages} %pdf-Dateien ganzseitig einbinden -- Alternative zu \includegraphics
%*******************************************

%********************************************
% "Kann"-Pakete, jeder muss selbst entscheiden ob er sie verwenden möchte
%\usepackage{booktabs} %Schöne Unterteilungsstriche für Tabellen (\toprule, \bottomrule, \midrule,...)
\usepackage{siunitx} %Formatierung für Zahlen, Einheiten, Winkel,...
\usepackage[babel]{csquotes} %Sprachenspezifische Zitierregeln und Anführungszeichen, Paket wird auch für biblatex benötigt
\usepackage[style=numeric-comp, sorting=none, alldates=short, abbreviate=false, maxcitenames=2, maxbibnames=99, eprint=false, hyperref=auto, backend=biber]{biblatex} %Literaturverzeichnis, Achtung: Für backend=biber muss man auch biber statt bibtex benutzen! Mehr Informationen: http://biblatex-biber.sourceforge.net/
%\usepackage{icomma} %Komma ist Dezimaltrenner im Mathemodus
%\usepackage{tabu} % Etwas komfortablerer Tabellensatz
%\usepackage{multicol} %Mehrspaltiger Satz
%\usepackage{pgfplots} %Erstellt Plots von Funktionen oder Messwerten
%********************************************

%********************************************
% Hyperref wird sehr empfohlen, muss aber als letztes Paket eingebunden werden
\ifPDFTeX
	\usepackage[hyperfootnotes=false]{hyperref}
\else
	\usepackage[unicode]{hyperref}
\fi
\usepackage[all]{hypcap} %Verweise auf Floats (Abb., Tabellen) zeigen nicht auf die Bildunterschrift sondern auf das obere Ende des Bildes
%********************************************

%*********** Seitenlayout, TUD-Design *****************************************************
\geometry{left=35mm, right=20mm, top=20mm, bottom=20mm, nomarginpar, includeall} %Seitenränder EMK-konform setzen
\setinstitutionlogo{Figures/EMK_Logo}
%\printpicturesize %Gibt den verfügbaren Platz für das Titelbild auf der Titelseite aus, praktisch um ein Bild zurechtzuschneiden
%****************************************************************************

%********** Kopf- und Fußzeilen ***********************************************
\makeatletter
% Kopf- und Fußzeile für den Hauptteil
\fancypagestyle{headings}{%
    \fancyhf{} %Alle Kopf- und Fusszeilen abschalten
    \fancyhead[L]{\TUD@indentbar[\headwidth]} %Kopfzeile oben links ist TUD-Identitätsbalken über die gesamte Breite
		\fancyhead[R]{} %Kopfzeile rechts leer
    \fancyfoot[L]{\footerfont\strut\\\Thesisname\ -- \Nachname} %TBD Fußzeile links
		\fancyfoot[C]{\tudrule[\headwidth]\footerfont\strut\\\nouppercase\leftmark} %Überkapitel in der Fußzeile zentriert
		\fancyfoot[R]{\footerfont\strut\\\thepage} %Seitenzahl in der Fußzeile außen
  }
% Kopf- und Fußzeile für Kapitelseiten
\fancypagestyle{chapterheadings}{%
    \fancyhf{} %Alle Kopf- und Fusszeilen abschalten
    \fancyhead[L]{\TUD@indentbar[\headwidth]} %TUD-Identitätsbalken
		\fancyhead[R]{}
    \fancyfoot[L]{\footerfont\strut\\\Thesisname\ -- \Nachname} %TBD Fußzeile links
		\fancyfoot[C]{\tudrule[\headwidth]\footerfont\strut\\\nouppercase\leftmark} %Überkapitel in der Fußzeile zentriert
		\fancyfoot[R]{\footerfont\strut\\\thepage} %Seitenzahl in der Fußzeile außen
  }
%Kopf- und Fußzeile für die Zusammenfassung
\fancypagestyle{zusammenfassung}{%
	\fancyhf{} %Alle Kopf- und Fusszeilen abschalten
	\fancyhead[L]{\TUD@indentbar[\headwidth]} %TUD-Identitätsbalken
	\fancyfoot[C]{\tudrule[\headwidth]\footerfont\strut\\\Thesisname\ -- \Nachname\\Institut EMK -- TU Darmstadt}
}

\@ifundefined{chapter}{}{ %falls \chapter existiert, wird das folgende ausgeführt
	\renewcommand*{\chapterpagestyle}{headings} %Kopf- und Fußzeilen auch auf Seiten, auf denen ein neues Kapitel beginnt
}
\makeatother
%****************************************************************************

%*************** Automatische Positionierung von Floatobjekten (Bildern, Tabellen, etc.)*************
% siehe http://projekte.dante.de/DanteFAQ/FloatPlatzierung
\makeatletter
\renewcommand{\fps@figure}{htbp} %Standardplatzierungsoption für figures auf htbp ändern
\renewcommand{\fps@table}{htbp} %Standardplatzierungsoption für tables auf htbp ändern
\makeatother
\setcounter{topnumber}           {3} %Maximale Anzahl Floatobjekte oben auf einer Seite (t), Standard 2
\setcounter{bottomnumber}        {1} %Maximale Anzahl Floatobjekte unten auf einer Seite (b), Standard 1
\setcounter{totalnumber}         {4} %Maximale Anzahl Floatobjekte gesamt auf einer Seite, Standard 3
\renewcommand{\floatpagefraction}{0.8} %Minimaler Seitenanteil für Platzierung page (p), Standard 0.5
\renewcommand{\topfraction}      {0.9} %Maximaler Seitenanteil der für Platzierung top (t) verwendet werden darf, Standard 0.7
\renewcommand{\bottomfraction}   {0.6} %Maximaler Seitenanteil der für Platzierung bottom (b) verwendet werden darf, Standard 0.3
\renewcommand{\textfraction}     {0.15} %Minimaler Seitenanteil Text auf einer Seite, Standard 0.2
%*******************************************************************************

%*************hyperref*************************************************
\hypersetup{pdfauthor={\Vorname \Nachname}}
\addto\extrasngerman{%
	\renewcommand{\appendixautorefname}{\appendixname}
	\renewcommand{\sectionautorefname}{Kapitel}
	\renewcommand{\subsectionautorefname}{Kapitel}
}
% Hack von Heiko Oberdieck in de.comp.text.tex, damit autoref bei Verweisen sowohl auf Kapitel als auch auf Unterkapitel im Anhang immer schreibt "siehe Anhang xy"
% Suchbegriff bei google: in:de.comp.text.tex Autoref im Anhang
\makeatletter
\@ifundefined{chapter}{}{
	\newcommand*{\theappendix}{\thechapter}
	\newcommand*{\theHappendix}{%
		\number\value{chapter}.\number\value{section}%
	}
	\let\org@hyper@makecurrent\hyper@makecurrent
	\def\string@section{section}
	\def\hyper@makecurrent#1{%
		\expandafter\ifx\@chapapp\appendixname
			\def\string@temp{#1}%
			\ifx\string@section\string@temp
				\org@hyper@makecurrent{appendix}%
			\else
				\org@hyper@makecurrent{#1}%
			\fi
		\else
			\org@hyper@makecurrent{#1}%
		\fi
	}
}
\makeatother
%**********************************************************************

%********** biblatex, csquotes ********************************************************
\MakeOuterQuote{"}
\DefineBibliographyStrings{ngerman}{
	andothers={et\ \addabbrvspace al\adddot},
	urlseen={abgerufen am}
	}
\renewcommand*{\mkbibnamelast}[1]{\textsc{#1}} %Autorennamen in Kapitälchen
\renewcommand*{\multinamedelim}{\addsemicolon\space} % Namen im Literaturverzeichnis mit Semikolon getrennt
\DeclareNameAlias{default}{last-first} % Bei Namen Nachname zuerst
% Nachnamen, dann Vornamen abgekürzt
%\DeclareNameFormat{labelname}{%
  %\usebibmacro{name:last-first}{#1}{#4}{#6}{#8}%
  %\usebibmacro{name:andothers}}
%\DeclareNameFormat{sortname}{%
  %\usebibmacro{name:last-first}{#1}{#4}{#6}{#8}%
  %\usebibmacro{name:andothers}} 
% Wesentlicher Titel immer kursiv. Buchtitel, Journaltitel, etc. in Anführungszeichen.
\DeclareFieldFormat{journaltitle}{\mkbibquote{#1\isdot}}
\DeclareFieldFormat{maintitle}{\mkbibquote{#1\isdot}}
\DeclareFieldFormat{booktitle}{\mkbibquote{#1\isdot}}
\DeclareFieldFormat[article]{title}{\mkbibemph{#1}}
\DeclareFieldFormat[inbook]{title}{\mkbibemph{#1}}
\DeclareFieldFormat[incollection]{title}{\mkbibemph{#1}}
\DeclareFieldFormat[inproceedings]{title}{\mkbibemph{#1}}
\DeclareFieldFormat[patent]{title}{\mkbibemph{#1}}
\DeclareFieldFormat[thesis]{title}{\mkbibemph{#1}} 
%****************************************************************************

%********* siunitx ***************************************************
\sisetup{redefine-symbols=false,
			detect-all,
			text-micro={{\normalfont\textmu}},
			math-micro={\mbox{\normalfont\textmu}},
}
\addto\extrasngerman{
	\sisetup{
	locale=DE
	}
}
%**********************************************************************